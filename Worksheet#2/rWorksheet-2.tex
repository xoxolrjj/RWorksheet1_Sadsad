% Options for packages loaded elsewhere
\PassOptionsToPackage{unicode}{hyperref}
\PassOptionsToPackage{hyphens}{url}
%
\documentclass[
]{article}
\usepackage{amsmath,amssymb}
\usepackage{iftex}
\ifPDFTeX
  \usepackage[T1]{fontenc}
  \usepackage[utf8]{inputenc}
  \usepackage{textcomp} % provide euro and other symbols
\else % if luatex or xetex
  \usepackage{unicode-math} % this also loads fontspec
  \defaultfontfeatures{Scale=MatchLowercase}
  \defaultfontfeatures[\rmfamily]{Ligatures=TeX,Scale=1}
\fi
\usepackage{lmodern}
\ifPDFTeX\else
  % xetex/luatex font selection
\fi
% Use upquote if available, for straight quotes in verbatim environments
\IfFileExists{upquote.sty}{\usepackage{upquote}}{}
\IfFileExists{microtype.sty}{% use microtype if available
  \usepackage[]{microtype}
  \UseMicrotypeSet[protrusion]{basicmath} % disable protrusion for tt fonts
}{}
\makeatletter
\@ifundefined{KOMAClassName}{% if non-KOMA class
  \IfFileExists{parskip.sty}{%
    \usepackage{parskip}
  }{% else
    \setlength{\parindent}{0pt}
    \setlength{\parskip}{6pt plus 2pt minus 1pt}}
}{% if KOMA class
  \KOMAoptions{parskip=half}}
\makeatother
\usepackage{xcolor}
\usepackage[margin=1in]{geometry}
\usepackage{color}
\usepackage{fancyvrb}
\newcommand{\VerbBar}{|}
\newcommand{\VERB}{\Verb[commandchars=\\\{\}]}
\DefineVerbatimEnvironment{Highlighting}{Verbatim}{commandchars=\\\{\}}
% Add ',fontsize=\small' for more characters per line
\usepackage{framed}
\definecolor{shadecolor}{RGB}{248,248,248}
\newenvironment{Shaded}{\begin{snugshade}}{\end{snugshade}}
\newcommand{\AlertTok}[1]{\textcolor[rgb]{0.94,0.16,0.16}{#1}}
\newcommand{\AnnotationTok}[1]{\textcolor[rgb]{0.56,0.35,0.01}{\textbf{\textit{#1}}}}
\newcommand{\AttributeTok}[1]{\textcolor[rgb]{0.13,0.29,0.53}{#1}}
\newcommand{\BaseNTok}[1]{\textcolor[rgb]{0.00,0.00,0.81}{#1}}
\newcommand{\BuiltInTok}[1]{#1}
\newcommand{\CharTok}[1]{\textcolor[rgb]{0.31,0.60,0.02}{#1}}
\newcommand{\CommentTok}[1]{\textcolor[rgb]{0.56,0.35,0.01}{\textit{#1}}}
\newcommand{\CommentVarTok}[1]{\textcolor[rgb]{0.56,0.35,0.01}{\textbf{\textit{#1}}}}
\newcommand{\ConstantTok}[1]{\textcolor[rgb]{0.56,0.35,0.01}{#1}}
\newcommand{\ControlFlowTok}[1]{\textcolor[rgb]{0.13,0.29,0.53}{\textbf{#1}}}
\newcommand{\DataTypeTok}[1]{\textcolor[rgb]{0.13,0.29,0.53}{#1}}
\newcommand{\DecValTok}[1]{\textcolor[rgb]{0.00,0.00,0.81}{#1}}
\newcommand{\DocumentationTok}[1]{\textcolor[rgb]{0.56,0.35,0.01}{\textbf{\textit{#1}}}}
\newcommand{\ErrorTok}[1]{\textcolor[rgb]{0.64,0.00,0.00}{\textbf{#1}}}
\newcommand{\ExtensionTok}[1]{#1}
\newcommand{\FloatTok}[1]{\textcolor[rgb]{0.00,0.00,0.81}{#1}}
\newcommand{\FunctionTok}[1]{\textcolor[rgb]{0.13,0.29,0.53}{\textbf{#1}}}
\newcommand{\ImportTok}[1]{#1}
\newcommand{\InformationTok}[1]{\textcolor[rgb]{0.56,0.35,0.01}{\textbf{\textit{#1}}}}
\newcommand{\KeywordTok}[1]{\textcolor[rgb]{0.13,0.29,0.53}{\textbf{#1}}}
\newcommand{\NormalTok}[1]{#1}
\newcommand{\OperatorTok}[1]{\textcolor[rgb]{0.81,0.36,0.00}{\textbf{#1}}}
\newcommand{\OtherTok}[1]{\textcolor[rgb]{0.56,0.35,0.01}{#1}}
\newcommand{\PreprocessorTok}[1]{\textcolor[rgb]{0.56,0.35,0.01}{\textit{#1}}}
\newcommand{\RegionMarkerTok}[1]{#1}
\newcommand{\SpecialCharTok}[1]{\textcolor[rgb]{0.81,0.36,0.00}{\textbf{#1}}}
\newcommand{\SpecialStringTok}[1]{\textcolor[rgb]{0.31,0.60,0.02}{#1}}
\newcommand{\StringTok}[1]{\textcolor[rgb]{0.31,0.60,0.02}{#1}}
\newcommand{\VariableTok}[1]{\textcolor[rgb]{0.00,0.00,0.00}{#1}}
\newcommand{\VerbatimStringTok}[1]{\textcolor[rgb]{0.31,0.60,0.02}{#1}}
\newcommand{\WarningTok}[1]{\textcolor[rgb]{0.56,0.35,0.01}{\textbf{\textit{#1}}}}
\usepackage{graphicx}
\makeatletter
\def\maxwidth{\ifdim\Gin@nat@width>\linewidth\linewidth\else\Gin@nat@width\fi}
\def\maxheight{\ifdim\Gin@nat@height>\textheight\textheight\else\Gin@nat@height\fi}
\makeatother
% Scale images if necessary, so that they will not overflow the page
% margins by default, and it is still possible to overwrite the defaults
% using explicit options in \includegraphics[width, height, ...]{}
\setkeys{Gin}{width=\maxwidth,height=\maxheight,keepaspectratio}
% Set default figure placement to htbp
\makeatletter
\def\fps@figure{htbp}
\makeatother
\setlength{\emergencystretch}{3em} % prevent overfull lines
\providecommand{\tightlist}{%
  \setlength{\itemsep}{0pt}\setlength{\parskip}{0pt}}
\setcounter{secnumdepth}{-\maxdimen} % remove section numbering
\ifLuaTeX
  \usepackage{selnolig}  % disable illegal ligatures
\fi
\IfFileExists{bookmark.sty}{\usepackage{bookmark}}{\usepackage{hyperref}}
\IfFileExists{xurl.sty}{\usepackage{xurl}}{} % add URL line breaks if available
\urlstyle{same}
\hypersetup{
  pdftitle={RWorksheets\#2},
  pdfauthor={Missy Key Sadsad},
  hidelinks,
  pdfcreator={LaTeX via pandoc}}

\title{RWorksheets\#2}
\author{Missy Key Sadsad}
\date{2023-09-28}

\begin{document}
\maketitle

R Worksheet for R programming

1.Create a vector using: Operator

\begin{enumerate}
\def\labelenumi{\alph{enumi}.}
\tightlist
\item
  Sequence from -5 to 5. Write the R code and its output. Describe its
  output.
\end{enumerate}

\begin{Shaded}
\begin{Highlighting}[]
\NormalTok{seq \textless{}{-} c({-}5:5)  }
\NormalTok{seq}
\NormalTok{{-}5 {-}4 {-}3 {-}2 {-}1  0  1  2  3  4  5 }
\NormalTok{The sequence displays the negative numbers from {-}5 then increases by 1 to positive 5 only}
\end{Highlighting}
\end{Shaded}

\begin{enumerate}
\def\labelenumi{\alph{enumi}.}
\setcounter{enumi}{1}
\tightlist
\item
  x \textless- 1:7. What will be the value of x?
\end{enumerate}

\begin{Shaded}
\begin{Highlighting}[]
\NormalTok{x \textless{}{-} 1:7}
\NormalTok{x}
\NormalTok{The value of x is: }
\NormalTok{1 2 3 4 5 6 7}
\end{Highlighting}
\end{Shaded}

2.Create a vector using seq() function

\begin{enumerate}
\def\labelenumi{\alph{enumi}.}
\tightlist
\item
  seq(1, 3, by=0.2) \# specify step size. Write the R script and its
  output. Describe the output.
\end{enumerate}

\begin{Shaded}
\begin{Highlighting}[]
\NormalTok{seq(1, 3, by = 0.2) }
\NormalTok{1.0 1.2 1.4 1.6 1.8 2.0 2.2 2.4 2.6 2.8 3.0 }
\NormalTok{It increases its value by 0.2 until it reaches its maximum value at 3}
\end{Highlighting}
\end{Shaded}

3.A factory has a census of its workers. There are 50 workers in total.

\begin{verbatim}
The following list shows their ages: 34, 28, 22, 36, 27, 18, 52, 39, 42, 29, 35, 31, 27, 22, 37, 34, 19, 
                                     20, 57, 49, 50, 37, 46, 25, 17, 37, 43, 53, 41, 51, 35, 24, 33, 41, 
                                     53, 40, 18, 44, 38, 41, 48, 27, 39, 19, 30, 61, 54, 58, 26, 18
\end{verbatim}

\begin{enumerate}
\def\labelenumi{\alph{enumi}.}
\tightlist
\item
  Access 3rd element, what is the value?
\end{enumerate}

\begin{Shaded}
\begin{Highlighting}[]
\NormalTok{\textgreater{}worker\_age[3] \#Its value is 22}
\end{Highlighting}
\end{Shaded}

\begin{enumerate}
\def\labelenumi{\alph{enumi}.}
\setcounter{enumi}{1}
\tightlist
\item
  Access 2nd and 4th element, what are the values?
\end{enumerate}

\begin{Shaded}
\begin{Highlighting}[]
\NormalTok{\textgreater{}worker\_age1 \textless{}{-} worker\_age[c(2,4)]}
\NormalTok{\textgreater{}worker\_age1  \#Its value is 28 \& 36 }
\end{Highlighting}
\end{Shaded}

\begin{enumerate}
\def\labelenumi{\alph{enumi}.}
\setcounter{enumi}{2}
\tightlist
\item
  Access all but the 4th and 12th element is not included. Write the R
  script and its output.
\end{enumerate}

\begin{Shaded}
\begin{Highlighting}[]
\NormalTok{\textgreater{} worker\_age2 \textless{}{-} worker\_age[{-}c(4,12)]}
\NormalTok{\textgreater{} worker\_age2}
\NormalTok{[[1]]}
\NormalTok{[1] 34}
\NormalTok{[[2]]}
\NormalTok{[1] 28}
\NormalTok{[[3]]}
\NormalTok{[1] 22}
\NormalTok{[[4]]}
\NormalTok{[1] 27}
\NormalTok{[[5]]}
\NormalTok{[1] 18}
\NormalTok{[[6]]}
\NormalTok{[1] 52}
\NormalTok{[[7]]}
\NormalTok{[1] 39}
\NormalTok{[[8]]}
\NormalTok{[1] 42}
\NormalTok{[[9]]}
\NormalTok{[1] 29}
\NormalTok{[[10]]}
\NormalTok{[1] 35}
\NormalTok{[[11]]}
\NormalTok{[1] 27}
\NormalTok{[[12]]}
\NormalTok{[1] 22}
\NormalTok{[[13]]}
\NormalTok{[1] 37}
\NormalTok{[[14]]}
\NormalTok{[1] 34}
\NormalTok{[[15]]}
\NormalTok{[1] 19}
\NormalTok{[[16]]}
\NormalTok{[1] 20}
\NormalTok{[[17]]}
\NormalTok{[1] 57}
\NormalTok{[[18]]}
\NormalTok{[1] 49}
\NormalTok{[[19]]}
\NormalTok{[1] 50}
\NormalTok{[[20]]}
\NormalTok{[1] 37}
\NormalTok{[[21]]}
\NormalTok{[1] 46}
\NormalTok{[[22]]}
\NormalTok{[1] 25}
\NormalTok{[[23]]}
\NormalTok{[1] 17}
\NormalTok{[[24]]}
\NormalTok{[1] 37}
\NormalTok{[[25]]}
\NormalTok{[1] 43}
\NormalTok{[[26]]}
\NormalTok{[1] 53}
\NormalTok{[[27]]}
\NormalTok{[1] 41}
\NormalTok{[[28]]}
\NormalTok{[1] 51}
\NormalTok{[[29]]}
\NormalTok{[1] 35}
\NormalTok{[[30]]}
\NormalTok{[1] 24}
\NormalTok{[[31]]}
\NormalTok{[1] 33}
\NormalTok{[[32]]}
\NormalTok{[1] 41}
\NormalTok{[[33]]}
\NormalTok{[1] 53}
\NormalTok{[[34]]}
\NormalTok{[1] 40}
\NormalTok{[[35]]}
\NormalTok{[1] 18}
\NormalTok{[[36]]}
\NormalTok{[1] 44}
\NormalTok{[[37]]}
\NormalTok{[1] 38}
\NormalTok{[[38]]}
\NormalTok{[1] 41}
\NormalTok{[[39]]}
\NormalTok{[1] 48}
\NormalTok{[[40]]}
\NormalTok{[1] 27}
\NormalTok{[[41]]}
\NormalTok{[1] 39}
\NormalTok{[[42]]}
\NormalTok{[1] 19}
\NormalTok{[[43]]}
\NormalTok{[1] 30}
\NormalTok{[[44]]}
\NormalTok{[1] 61}
\NormalTok{[[45]]}
\NormalTok{[1] 54}
\NormalTok{[[46]]}
\NormalTok{[1] 58}
\NormalTok{[[47]]}
\NormalTok{[1] 26}
\NormalTok{[[48]]}
\NormalTok{[1] 18}
\end{Highlighting}
\end{Shaded}

4.Create a vector x \textless- c(``first''=3, ``second''=0,
``third''=9). Then named the vector, names(x).

\begin{Shaded}
\begin{Highlighting}[]
\NormalTok{num4 \textless{}{-} c("first"=3, "second"=0, "third"=9)}
 
\end{Highlighting}
\end{Shaded}

\begin{enumerate}
\def\labelenumi{\alph{enumi}.}
\tightlist
\item
  The output displays only the ``first'' and ``third'' variables using
  array. Describe the output.
\end{enumerate}

\begin{Shaded}
\begin{Highlighting}[]
\NormalTok{num4[c("first", "third")] }
\NormalTok{num4\#The output displays only the "first" and "third" variables using array.}
\NormalTok{first third }
\NormalTok{    3     9}
\end{Highlighting}
\end{Shaded}

\begin{enumerate}
\def\labelenumi{\alph{enumi}.}
\setcounter{enumi}{1}
\tightlist
\item
  Write the code and its output.
\end{enumerate}

\begin{Shaded}
\begin{Highlighting}[]
\OperatorTok{\textgreater{}}\NormalTok{num4 }\OperatorTok{\textless{}{-}}\NormalTok{ c}\OperatorTok{(}\StringTok{"first"}\OperatorTok{=}\DecValTok{3}\OperatorTok{,} \StringTok{"second"}\OperatorTok{=}\DecValTok{0}\OperatorTok{,} \StringTok{"third"}\OperatorTok{=}\DecValTok{9}\OperatorTok{)}
\NormalTok{num4 \#first second  third }
\NormalTok{     \#  }\DecValTok{3}      \DecValTok{0}      \DecValTok{9} 
\OperatorTok{\textgreater{}}\NormalTok{num4[c}\OperatorTok{(}\StringTok{"first"}\OperatorTok{,} \StringTok{"third"}\OperatorTok{)}\NormalTok{]}

\NormalTok{num4 \#first third }
\NormalTok{     \#    }\DecValTok{3}     \DecValTok{9}
\end{Highlighting}
\end{Shaded}

5.Create a sequence x from -3:2.

\begin{enumerate}
\def\labelenumi{\alph{enumi}.}
\tightlist
\item
  Modify 2nd element and change it to 0; x{[}2{]} \textless- 0 x
  Describe the output.
\end{enumerate}

\begin{Shaded}
\begin{Highlighting}[]
\NormalTok{num5[2] \textless{}{-} 0}
\NormalTok{num5  \#The second element in the array was changed to 0 and the result is when it is sequenced, the {-}2 }
\NormalTok{      was changed to 0}
\end{Highlighting}
\end{Shaded}

\begin{enumerate}
\def\labelenumi{\alph{enumi}.}
\setcounter{enumi}{1}
\tightlist
\item
  Write the code and its output.
\end{enumerate}

\begin{Shaded}
\begin{Highlighting}[]
\NormalTok{num5[2] \textless{}{-} 0}
\NormalTok{num5}
\NormalTok{[1] {-}3  0 {-}1  0  1  2}
\end{Highlighting}
\end{Shaded}

6.The following data shows the diesel fuel purchased by Mr.~Cruz.

Month

Jan

Feb

March

Apr

May

June

Price per liter (PhP)

52.50

57.25

60.00

65.00

74.25

54.00

Purchase--quantity(Liters)

25

30

40

50

10

45

\begin{enumerate}
\def\labelenumi{\alph{enumi}.}
\tightlist
\item
  Create a data frame for month, price per liter (php) and
  purchase-quantity (liter). Write the R scripts and its output.
\end{enumerate}

\begin{Shaded}
\begin{Highlighting}[]
\NormalTok{\textgreater{}month \textless{}{-} c("Jan", "Feb", "March", "Apr", "May", "June")}
\NormalTok{\textgreater{}price\_per\_liter \textless{}{-} c(52.50,57.25,60.00,65.00,74.25,54.00)}
\NormalTok{\textgreater{}purchase\_quantity \textless{}{-} c(25,30,40,50,10,45)}
\NormalTok{\textgreater{}data.frame \textless{}{-} data.frame(month,  price\_per\_liter,  purchase\_quantity)}
\NormalTok{\textgreater{}data.frame}
\NormalTok{month price\_per\_liter purchase\_quantity}
\NormalTok{1Jan           52.50                25}
\NormalTok{2Feb           57.25                30}
\NormalTok{3March         60.00                40}
\NormalTok{4Apr           65.00                50}
\NormalTok{5May           74.25                10}
\NormalTok{6June          54.00                45}
\end{Highlighting}
\end{Shaded}

\begin{enumerate}
\def\labelenumi{\alph{enumi}.}
\setcounter{enumi}{1}
\tightlist
\item
  What is the average fuel expenditure of Mr.~Cruz from Jan to June?
  Note: Use 'weighted.mean(liter, purchase)'. Write the R scripts and
  its output.
\end{enumerate}

\begin{Shaded}
\begin{Highlighting}[]
\NormalTok{The avg fuel expenditure of Mr. Cruz from Jan to June is 59.2625}
\NormalTok{weighted.mean(price\_per\_liter,purchase\_quantity)}
\NormalTok{[1] 59.2625}
\end{Highlighting}
\end{Shaded}

7.R has actually lots of built-in datasets. For example, the rivers data
``gives the lengths (in miles) of 141 ``major'' rivers in North America,
as compiled by the US Geological Survey''.

\begin{enumerate}
\def\labelenumi{\alph{enumi}.}
\tightlist
\item
  Type ``rivers'' in your R console. Create a vector data with 7
  elements, containing the number of elements (length) in rivers, their
  sum (sum), mean (mean), median(median), variance(var), standard
  deviation(sd), minimum (min) and maximum (max).
\end{enumerate}

\begin{Shaded}
\begin{Highlighting}[]
\NormalTok{data \textless{}{-} c(length(rivers), sum(rivers), mean(rivers), median(rivers), var(rivers), sd(rivers), min(rivers), max(rivers))}
\end{Highlighting}
\end{Shaded}

\begin{enumerate}
\def\labelenumi{\alph{enumi}.}
\setcounter{enumi}{1}
\tightlist
\item
  What are the results?
\end{enumerate}

\begin{Shaded}
\begin{Highlighting}[]
\NormalTok{[1]141.0000  83357.0000    591.1844    425.0000 243908.4086    493.8708    135.0000   3710.0000 }
\end{Highlighting}
\end{Shaded}

\begin{enumerate}
\def\labelenumi{\alph{enumi}.}
\setcounter{enumi}{2}
\tightlist
\item
  Write the R scripts and its outputs.
\end{enumerate}

\begin{Shaded}
\begin{Highlighting}[]
\NormalTok{data \textless{}{-} c(length(rivers), sum(rivers), mean(rivers), median(rivers), var(rivers),}
\NormalTok{          sd(rivers), min(rivers), max(rivers))}
\NormalTok{data}
\NormalTok{[1]141.0000  83357.0000    591.1844    425.0000 243908.4086    493.8708    135.0000   3710.0000}
\end{Highlighting}
\end{Shaded}

8.The table below gives the 25 most powerful celebrities and their
annual pay as ranked by the editions of Forbes magazine and as listed on
the Forbes.com website.

\begin{Shaded}
\begin{Highlighting}[]
\NormalTok{ power\_ranking   celebtrity\_name   pay}
\NormalTok{            1        Tom Cruise    67}
\NormalTok{            2     Rolling Stone    90}
\NormalTok{            3     Oprah Winfrey    225}
\NormalTok{            4                U2    110}
\NormalTok{            5       Tiger Woods    90}
\NormalTok{            6  Steven Spielberg    332}
\NormalTok{            7      Howard Stern    302}
\NormalTok{            8           50 Cent    41}
\NormalTok{            9  Cast of Sopranos    52}
\NormalTok{            10         Dan Brown   88}
\NormalTok{            11 Bruce Springsteen   55}
\NormalTok{            12      Donald Trump   44}
\NormalTok{            13      Muhammad Ali   55}
\NormalTok{            14    Paul McCartney   40}
\NormalTok{            15      George Lucas   233}
\NormalTok{            16        Elton John   34}
\NormalTok{            17   David Letterman   40}
\NormalTok{            18    Phil Mickelson   47}
\NormalTok{            19       J.K Rowling   75}
\NormalTok{            20        Bradd Pitt   25}
\NormalTok{            21     Peter Jackson   39}
\NormalTok{            22   Dr. Phil McGraw   45}
\NormalTok{            23       Jay   Lenon   32}
\NormalTok{            24       Celine Dion   40}
\NormalTok{            25       Kobe Bryant   31}
\end{Highlighting}
\end{Shaded}

\begin{enumerate}
\def\labelenumi{\alph{enumi}.}
\tightlist
\item
  Create vectors according to the above table. Write the R scripts and
  its output.
\end{enumerate}

\begin{Shaded}
\begin{Highlighting}[]
\NormalTok{power\_ranking   \textless{}{-} c(1:25)}
\NormalTok{celebtrity\_name \textless{}{-} c("Tom Cruise", "Rolling Stone", "Oprah Winfrey", "U2", "Tiger Woods", "Steven Spielberg", }
\NormalTok{                     "Howard Stern", "50 Cent", "Cast of Sopranos", "Dan Brown", "Bruce Springsteen", }
\NormalTok{                     "Donald Trump","Muhammad Ali", "Paul McCartney", "George Lucas", "Elton John", }
\NormalTok{                     "David Letterman","Phil Mickelson","J.K Rowling", "Bradd Pitt", "Peter Jackson", }
\NormalTok{                     "Dr. Phil McGraw", "Jay Lenon","Celine Dion", "Kobe Bryant" )}
\NormalTok{pay             \textless{}{-} c(67,90,225,110,90,332,302,41,52,88,55,44,55,40,233,34,40,47,75,25,39,45,32,40,31)}
\end{Highlighting}
\end{Shaded}

\begin{enumerate}
\def\labelenumi{\alph{enumi}.}
\setcounter{enumi}{1}
\tightlist
\item
  Modify the power ranking and pay of J.K. Rowling. Change power ranking
  to 15 and pay to 90. Write the R scripts and its output.
\end{enumerate}

\begin{Shaded}
\begin{Highlighting}[]
\NormalTok{power\_ranking[19] \textless{}{-} 15}
\NormalTok{power\_ranking }
\NormalTok{[1]  1  2  3  4  5  6  7  8  9 10 11 12 13 14 15 16 17 18 15 20 21 22 23 24 25}
\NormalTok{pay[19] \textless{}{-} 90}
\NormalTok{pay }
\NormalTok{[1]  67  90 225 110  90 332 302  41  52  88  55  44  55  40 233  34  40  47  90  25  39  45  32  40  31}
\end{Highlighting}
\end{Shaded}

\begin{enumerate}
\def\labelenumi{\alph{enumi}.}
\setcounter{enumi}{2}
\tightlist
\item
  Create an excel file from the table above and save it as csv
  file(PowerRanking). Import the csv file into the RStudio. What is the
  R script?
\end{enumerate}

\begin{Shaded}
\begin{Highlighting}[]
\NormalTok{csv\_file \textless{}{-} "PowerRanking.csv"}
\NormalTok{write.csv(PowerRanking, file = csv\_file)}
\NormalTok{PowerRankingCSV \textless{}{-} read.csv("PowerRanking.csv")}
\end{Highlighting}
\end{Shaded}

\begin{enumerate}
\def\labelenumi{\alph{enumi}.}
\setcounter{enumi}{3}
\tightlist
\item
  Access the rows 10 to 20 and save it as Ranks.RData. Write the R
  script and its output.
\end{enumerate}

\begin{Shaded}
\begin{Highlighting}[]

\NormalTok{Power\_Ranking \textless{}{-} PowerRankingCSV[10:20,]}
\NormalTok{Power\_Ranking}
\NormalTok{\#                        power\_ranking   celebtrity\_name      pay}
\NormalTok{\#           10            10                 Dan Brown        88}
\NormalTok{\#           11            11              Bruce Springsteen   55}
\NormalTok{\#           12            12                 Donald Trump     44}
\NormalTok{\#           13            13                Muhammad Ali      55}
\NormalTok{\#           14            14              Paul McCartney      40}
\NormalTok{\#           15            15             George Lucas        233}
\NormalTok{\#           16            16               Elton John         34}
\NormalTok{\#           17            17             David Letterman      40}
\NormalTok{\#           18            18              Phil Mickelson      47}
\NormalTok{\#           19            19               J.K Rowling        75}
\NormalTok{\#           20            20                Bradd Pitt        25}
\NormalTok{save(Power\_Ranking, file = "Ranks.RData")}
\NormalTok{load("Ranks.RData")}
\end{Highlighting}
\end{Shaded}

\begin{enumerate}
\def\labelenumi{\alph{enumi}.}
\setcounter{enumi}{4}
\tightlist
\item
  Describe its output.
\end{enumerate}

\begin{Shaded}
\begin{Highlighting}[]
\NormalTok{The PowerRank result was change to 10 to 20 elements.}
\end{Highlighting}
\end{Shaded}

9.Download the Hotels-Vienna \url{https://tinyurl.com/Hotels-Vienna}

\begin{enumerate}
\def\labelenumi{\alph{enumi}.}
\tightlist
\item
  Import the excel file into your RStudio. What is the R script?
\end{enumerate}

\begin{Shaded}
\begin{Highlighting}[]
\NormalTok{library(readxl)}
\NormalTok{hotels\_vienna \textless{}{-} read\_excel("hotels{-}vienna.xlsx")}
\NormalTok{View(hotels\_vienna)}
\NormalTok{hotels\_vienna}
\end{Highlighting}
\end{Shaded}

\begin{enumerate}
\def\labelenumi{\alph{enumi}.}
\setcounter{enumi}{1}
\tightlist
\item
  How many dimensions does the dataset have? What is the R script? WHat
  is its output?
\end{enumerate}

\begin{Shaded}
\begin{Highlighting}[]
\NormalTok{dim(hotels\_vienna)  }
\NormalTok{[1] 428  24}
\end{Highlighting}
\end{Shaded}

\begin{enumerate}
\def\labelenumi{\alph{enumi}.}
\setcounter{enumi}{2}
\tightlist
\item
  Select columns country, neighbourhood, price, stars,
  accomodation\_type, and ratings. Write the R script.
\end{enumerate}

\begin{Shaded}
\begin{Highlighting}[]
\NormalTok{col \textless{}{-} colnames(hotels\_vienna)}
\NormalTok{col}
\NormalTok{colnames1 \textless{}{-} col[c(1,6,7,8,22,24)]}
\NormalTok{colnames1 }
\NormalTok{[1] "country"            "neighbourhood"      "price"              "city"            "accommodation\_type"}
\NormalTok{[6] "rating"}
\end{Highlighting}
\end{Shaded}

\begin{enumerate}
\def\labelenumi{\alph{enumi}.}
\setcounter{enumi}{3}
\tightlist
\item
  Save the data as **new.RData to your RStudio. Write the R script.
\end{enumerate}

\begin{Shaded}
\begin{Highlighting}[]
\NormalTok{save(hotels\_vienna, file = "new.RData")}
\NormalTok{new \textless{}{-}load("new.RData")}
\NormalTok{View(new)}
\end{Highlighting}
\end{Shaded}

\begin{enumerate}
\def\labelenumi{\alph{enumi}.}
\setcounter{enumi}{4}
\tightlist
\item
  Describe its output.
\end{enumerate}

\begin{Shaded}
\begin{Highlighting}[]
\NormalTok{head(hotels\_vienna, 6) }
\NormalTok{\# country city\_actual rating\_count center1label center2label neighbourhood price city   stars ratingta}
\NormalTok{\# \textless{}chr\textgreater{}   \textless{}chr\textgreater{}       \textless{}chr\textgreater{}        \textless{}chr\textgreater{}        \textless{}chr\textgreater{}        \textless{}chr\textgreater{}         \textless{}dbl\textgreater{} \textless{}chr\textgreater{}  \textless{}dbl\textgreater{} \textless{}chr\textgreater{}   }
\NormalTok{\# 1 Austria Vienna      36           City centre  Donauturm    17. Hernals      81 Vienna     4 4.5     }
\NormalTok{\# 2 Austria Vienna      189          City centre  Donauturm    17. Hernals      81 Vienna     4 3.5     }
\NormalTok{\# 3 Austria Vienna      53           City centre  Donauturm    Alsergrund       85 Vienna     4 3.5     }
\NormalTok{\# 4 Austria Vienna      55           City centre  Donauturm    Alsergrund       83 Vienna     3 4       }
\NormalTok{\# 5 Austria Vienna      33           City centre  Donauturm    Alsergrund       82 Vienna     4 3.5     }
\NormalTok{\# 6 Austria Vienna      25           City centre  Donauturm    Alsergrund      229 Vienna     5 4.5  }
\NormalTok{tail(hotels\_vienna, 6L)}
\NormalTok{\# 1 Austria Vienna      53           City centre  Donauturm    Wieden           73 Vienna   3   3       }
\NormalTok{\# 2 Austria Vienna      2            City centre  Donauturm    Wieden          109 Vienna   3   3       }
\NormalTok{\# 3 Austria Vienna      145          City centre  Donauturm    Wieden          185 Vienna   5   4       }
\NormalTok{\# 4 Austria Vienna      112          City centre  Donauturm    Wieden          100 Vienna   4   4.5     }
\NormalTok{\# 5 Austria Vienna      169          City centre  Donauturm    Wieden           58 Vienna   3   3       }
\NormalTok{\# 6 Austria Vienna      80           City centre  Donauturm    Wieden          110 Vienna   3.5 NA   }
\end{Highlighting}
\end{Shaded}

10.Create a list of ten (10) vegetables you ate during your lifetime. If
none, just list down.

\begin{enumerate}
\def\labelenumi{\alph{enumi}.}
\tightlist
\item
  Write the R scripts and its output.
\end{enumerate}

\begin{Shaded}
\begin{Highlighting}[]
\NormalTok{vegetables \textless{}{-} list("cabbage", "carrot", "spinach","potato", "garlic", "corn", "onion", "tomato", "eggplant", "cucumber")}
\NormalTok{vegetables}
\NormalTok{[[1]]}
\NormalTok{[1] "cabbage"}

\NormalTok{[[2]]}
\NormalTok{[1] "carrot"}

\NormalTok{[[3]]}
\NormalTok{[1] "spinach"}

\NormalTok{[[4]]}
\NormalTok{[1] "potato"}

\NormalTok{[[5]]}
\NormalTok{[1] "garlic"}

\NormalTok{[[6]]}
\NormalTok{[1] "corn"}

\NormalTok{[[7]]}
\NormalTok{[1] "onion"}

\NormalTok{[[8]]}
\NormalTok{[1] "tomato"}

\NormalTok{[[9]]}
\NormalTok{[1] "eggplant"}

\NormalTok{[[10]]}
\NormalTok{[1] "cucumber"}
\end{Highlighting}
\end{Shaded}

\begin{enumerate}
\def\labelenumi{\alph{enumi}.}
\setcounter{enumi}{1}
\tightlist
\item
  Add 2 additional vegetables after the last vegetables in the list.
  What is the R script and its output?
\end{enumerate}

\begin{Shaded}
\begin{Highlighting}[]
\NormalTok{addVegetables \textless{}{-} c(vegetables, "mushroom", "ginger")}
\NormalTok{addVegetables }
\NormalTok{[[1]]}
\NormalTok{[1] "cabbage"}

\NormalTok{[[2]]}
\NormalTok{[1] "carrot"}

\NormalTok{[[3]]}
\NormalTok{[1] "spinach"}

\NormalTok{[[4]]}
\NormalTok{[1] "potato"}

\NormalTok{[[5]]}
\NormalTok{[1] "garlic"}

\NormalTok{[[6]]}
\NormalTok{[1] "corn"}

\NormalTok{[[7]]}
\NormalTok{[1] "onion"}

\NormalTok{[[8]]}
\NormalTok{[1] "tomato"}

\NormalTok{[[9]]}
\NormalTok{[1] "eggplant"}

\NormalTok{[[10]]}
\NormalTok{[1] "cucumber"}

\NormalTok{[[11]]}
\NormalTok{[1] "mushroom"}

\NormalTok{[[12]]}
\NormalTok{[1] "ginger"}
\end{Highlighting}
\end{Shaded}

\begin{enumerate}
\def\labelenumi{\alph{enumi}.}
\setcounter{enumi}{2}
\tightlist
\item
  Add 4 additional vegetables after index 5. How many datapoints does
  your vegetable list have? What is the R script and its output?
\end{enumerate}

\begin{Shaded}
\begin{Highlighting}[]
\NormalTok{addVegetables4 \textless{}{-} append(addVegetables, c("lettuce", "zucchini", "radish", "bell pepper"), after = 5)}
\NormalTok{addVegetables4}
\NormalTok{[[1]]}
\NormalTok{[1] "cabbage"}

\NormalTok{[[2]]}
\NormalTok{[1] "carrot"}

\NormalTok{[[3]]}
\NormalTok{[1] "spinach"}

\NormalTok{[[4]]}
\NormalTok{[1] "potato"}

\NormalTok{[[5]]}
\NormalTok{[1] "garlic"}

\NormalTok{[[6]]}
\NormalTok{[1] "lettuce"}

\NormalTok{[[7]]}
\NormalTok{[1] "zucchini"}

\NormalTok{[[8]]}
\NormalTok{[1] "radish"}

\NormalTok{[[9]]}
\NormalTok{[1] "bell pepper"}

\NormalTok{[[10]]}
\NormalTok{[1] "corn"}

\NormalTok{[[11]]}
\NormalTok{[1] "onion"}

\NormalTok{[[12]]}
\NormalTok{[1] "tomato"}

\NormalTok{[[13]]}
\NormalTok{[1] "eggplant"}

\NormalTok{[[14]]}
\NormalTok{[1] "cucumber"}

\NormalTok{[[15]]}
\NormalTok{[1] "mushroom"}

\NormalTok{[[16]]}
\NormalTok{[1] "ginger"}
\NormalTok{num\_vegetables \textless{}{-} length(addVegetables4)}
\NormalTok{num\_vegetables }
\NormalTok{[1] 16}
\end{Highlighting}
\end{Shaded}

\begin{enumerate}
\def\labelenumi{\alph{enumi}.}
\setcounter{enumi}{3}
\tightlist
\item
  Remove the vegetables in index 5, 10, and 15. How many vegetables were
  left? Write the codes and its output.
\end{enumerate}

\begin{Shaded}
\begin{Highlighting}[]
\NormalTok{ddVegetables5 \textless{}{-} addVegetables4[{-}c(5,10,15)]}
\NormalTok{addVegetables5 }
\NormalTok{[[1]]}
\NormalTok{[1] "cabbage"}

\NormalTok{[[2]]}
\NormalTok{[1] "carrot"}

\NormalTok{[[3]]}
\NormalTok{[1] "spinach"}

\NormalTok{[[4]]}
\NormalTok{[1] "potato"}

\NormalTok{[[5]]}
\NormalTok{[1] "lettuce"}

\NormalTok{[[6]]}
\NormalTok{[1] "zucchini"}

\NormalTok{[[7]]}
\NormalTok{[1] "radish"}

\NormalTok{[[8]]}
\NormalTok{[1] "bell pepper"}

\NormalTok{[[9]]}
\NormalTok{[1] "onion"}

\NormalTok{[[10]]}
\NormalTok{[1] "tomato"}

\NormalTok{[[11]]}
\NormalTok{[1] "eggplant"}

\NormalTok{[[12]]}
\NormalTok{[1] "cucumber"}

\NormalTok{[[13]]}
\NormalTok{[1] "ginger"}
\NormalTok{num\_addVegetables5 \textless{}{-} length(addVegetables5)}
\NormalTok{num\_addVegetables5 }
\NormalTok{[1] 13}
\end{Highlighting}
\end{Shaded}


\end{document}
